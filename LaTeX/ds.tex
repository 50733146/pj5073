\chapter{分散式系統}
\renewcommand{\baselinestretch}{10} %設定行距
\section{前言}
\par
\renewcommand{\baselinestretch}{1} %設定行距
\twelve \qquad 要合作就必須溝通與協調,俗話說合作就是力量大,但是溝通與協調不僅僅是一種本能,更是一種能力。這就是分散式運算的問題來源,在分散式環境中,合作對象是分散各地的電腦,在這些電腦透過網路連接在一起,每台電腦都能讀立運行,且同時也能藉由分散式系統來進行合作,若以一個基層人員而言,需要將上級、同級、下級,甚至客戶,的每一件事項在對應的部分做串聯,甚至獨立解決,而產生一加一大於二的成效。
\par
\renewcommand{\baselinestretch}{20} %設定行距
\section{分散式系統的基本架構}
\par
\renewcommand{\baselinestretch}{1} %設定行距
\twelve \qquad 
\par
\renewcommand{\baselinestretch}{20} %設定行距

\section{事件的排序問題}
\par
\renewcommand{\baselinestretch}{1} %設定行距
\twelve \qquad 
\par}